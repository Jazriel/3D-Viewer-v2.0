\capitulo{5}{Aspectos relevantes del desarrollo del proyecto}

A continuación ilustramos consideraciones de importancia para el correcto progreso del proyecto en sus diferentes partes (visualización de los modelos, procesado de los mismos y seguridad).

\section{Iluminación, materiales y colores}
Aunque a priori la iluminación pueda parecer algo trivial si observamos el proyecto de forma general, es un aspecto que exige documentarse a fondo, además de requerir bastante ensayo y error. Refiriéndonos al código de \textit{three.js}, cada material tiene unas reacciones diferentes con la iluminación, y hemos pasado por varias configuraciones hasta llegar a la que tenemos actualmente.

El formato de fichero de los modelos 3D que empleábamos en el prototipo inicial (\textit{STL}) no podía soportar de forma nativa el color, así que la prioridad fue elegir un material fácil de configurar y que quedase bien; por lo que tras algunas pruebas se establece el \texttt{MeshPhongMaterial} como el más adecuado.
En conjunción con este material, emplearemos una luz de punto algo potente que enfoque la escena ubicada en la misma posición que la cámara y una luz de fondo tenue. Sin esta combinación no aparecen sombras, y por lo tanto no se aprecian las profundidades.

Justo al inicio de la migración a \textit{PLY}, la mayor urgencia era comprobar que el cargador o \textit{parser} de dicho formato funcionase de forma adecuada con algunos ejemplos, por lo que dejamos al constructor del objeto escoger el material por defecto, siendo éste el \texttt{MeshBasicMaterial}.

Con vistas a futuro, necesitábamos escoger, esta vez si, el material más realista que pudiésemos tener, y teniendo en cuenta la información que es capaz de albergar el formato, escogimos el \texttt{MeshStandardMaterial}. ¿Por qué éste y no el \texttt{MeshPhysicalMaterial}, si es físicamente más correcto?
La respuesta es fácil: \textit{PLY} no soporta la única característica extra que este material nos aporta, la reflectividad. Sin embargo, y aunque sabemos que el formato empleado en este momento soporta color, no utilizamos el color del modelo sino uno elegido de forma arbitraria.

Finalmente, seleccionamos el color del propio modelo y nos damos cuenta de que necesitamos una configuración de iluminación totalmente diferente cuando importamos el color. Ahora, si el modelo no tiene color propio, le asignaremos uno <<predeterminado>> además de usar la iluminación anterior.
Sin embargo, los modelos de color requieren una iluminación muy diferente, con una luz de fondo muy potente para dar claridad a los colores y una iluminación de punto bastante débil para evitar ver reflejos.

\section{Los números en coma flotante (\textit{float})}
Durante la realización del análisis de los modelos (ver sección \ref{sec:analisis-modelos}) hemos apreciado una ligera discrepancia entre los datos en coma flotante existentes en un archivo de entrada y los datos que hemos parseado con el analizador de modelos.

Al leer los datos y almacenarlos en memoria, surgen nuevos dígitos decimales además de los ya existentes, lo cual podríamos entender que altera el resultado; pues nada más lejos de la realidad. Observando la forma en la que los computadores almacenan los números en coma flotante~\cite{wiki:float-numbers}, dicho fenómeno es algo común. De hecho, la única forma en la que se pudiera alterar el resultado sería que los datos fuesen interpretados de forma diferente por el receptor. Esto no es posible, ya que la precisión con la que se leen los datos es la especificada en la definición del estándar del formato.

Para cerciorarnos de que nuestro \textit{parser} no interpreta de forma diferente a la del visor debemos comprobar que, efectivamente, este lee los mismos números. Sin mayor dilación, realizaremos una breve comparativa.

En la figura~\ref{fig:vertice-source} vemos cómo son los números en el archivo original, y estos tienen solamente 6 decimales. Después, podemos observar en la figura~\ref{fig:vertice-python} cómo son una vez los ha leído nuestro analizador de modelos. Finalmente, en la figura~\ref{fig:vertice-web-browser} observamos que el navegador interpreta los números de igual forma a \textit{Python}.

\imagen{vertice-source}{Cómo son los números en el archivo en realidad.}
\imagen{vertice-python}{Cómo los interpreta nuestro lector en \textit{Python}.}
\imagen{vertice-web-browser}{Cómo los almacena en la geometría del objeto three.js.}

¿Cómo podemos concluir entonces que nuestro programa no alterará los datos? Además de que en teoría empleamos el mismo formato, aquí tenemos la prueba de que, tanto nuestro lector como \textit{three.js} <<deforman>> los números añadiendo exactamente los mismos decimales.

\section{Seguridad en la subida de archivos}
La seguridad es un aspecto importante en cualquier aplicación, pero lo es más todavía en una \textit{web}, dado que estará más expuesta a todo tipo de ataques malintencionados. Un importante punto de interacción con algunos usuarios es el formulario de subida, aparentemente inofensivo pero que nos expone a diferentes vulnerabilidades. Una de las más conocidas, el \textit{<<Cross-Site Scripting>>} (\textit{XSS}~\cite{wiki:XSS}\cite{flask-docs:XSS}), no la deberíamos sufrir porque no tenemos la necesidad de almacenar código ejecutable de los usuarios en el servidor (como \textit{php}, por ejemplo), y además porque hemos filtrado los tipos de archivos que serán admitidos en el servidor. Conseguimos dicho filtrado añadiendo solamente las extensiones de archivo estrictamente necesarias en el constructor del selector de fichero en los formularios.

Podemos además sufrir otro tipo de ataque mediante la subida de archivos. Si se intenta subir un archivo cuyo nombre contenga caracteres que lo hagan interpretable como una ruta y no solamente como un fichero, corremos el peligro de que, con la ruta adecuada podamos sobrescribir o añadir archivos ajenos al directorio de subida. Aunque esto ya por sí solo puede causar muchos problemas, si además es un archivo ejecutable, nuestro formulario podría ser una potencial entrada a código malicioso. Para evitar esta problemática, \textit{Flask} pone a nuestra disposición la función \texttt{secure\_filename}\footnote{\url{http://werkzeug.pocoo.org/docs/0.12/utils/\#werkzeug.utils.secure_filename}}, que se encarga de traducir esos posibles caracteres problemáticos en otros inofensivos.

\section{Congreso <<EDULEARN>>}
Merece la pena destacar que este trabajo ha sido aceptado para ser presentado en el congreso <<EDULEARN>>\footnote{\url{https://iated.org/edulearn/}} bajo el título \textit{<<ON-LINE 3D VIEWER FOR EVALUATION OF BIOLOGICAL ANTHROPOLOGY STUDENTS>>}\footnote{\url{https://iated.org/concrete3/view_abstract.php?paper_id=57777}}. Podremos encontrar una copia del \textit{paper} en el disco entregado.