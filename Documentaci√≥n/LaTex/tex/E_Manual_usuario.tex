\apendice{Documentación de usuario}

\section{Introducción}
En esta sección vamos a explicar los diferentes aspectos involucrados en el uso de la aplicación.

\section{Requisitos de usuarios}
La parte del visor funciona en Firefox 50 y posteriores, Safari 10.1 y superior, y Chrome en versiones superiores a la 58.
Se requiere de conexión a Internet, además de estar autorizado a usar la aplicación.

\section{Instalación}
Suponemos que el usuario final tendrá un navegador web disponible, así que no necesitará instalar nada. No obstante, garantizamos la funcionalidad con Firefox\footnote{\url{https://support.mozilla.org/es/products/firefox/install-and-update-firefox}}.

\section{Manual del usuario}
A continuación explicaremos paso a paso cómo utilizar las funciones disponibles en la aplicación.

También definiremos algunos términos para conseguir una mejor comprensión de las instrucciones:
\begin{itemize}
	\item Barra de navegación: aparece en la parte superior de casi todas las vistas, y nos sirve para navegar por los diferentes elementos de la interfaz. Nos permite acceder al almacén de modelos, subir otros nuevos o cerrar sesión.
	\item Modelo: versión digital de cada elemento físico que vamos a emplear (restos óseos, fósiles, etc).
	\item Vista: cada una de las interfaces que es fundamentalmente diferente, en nuestro caso tenemos una página de \textit{login}, otra para subir nuevos modelos, otra más para previsualizar aquellos disponibles y otra para subir otros nuevos.
	\item Anotación: elemento gráfico que permite etiquetar un punto o zona del modelo mediante una esfera pegada al mismo. La etiqueta aparece en el menú de anotaciones sobre la barra de herramientas.
	\item Medida: elemento gráfico que permite conocer la distancia lineal entre dos puntos del modelos, además de poner un nombre concreto a dicha medida. En el modelo, se encuentra representada a través de dos esferas y una línea que las une. Aparecen sobre la barra de herramientas (concretamente sobre el menú de medidas) su etiqueta y medida.
	\item Barra de herramientas: aparece en la parte izquierda en la vista del visor, y permite realizar acciones con anotaciones y medidas. Nos da acceso a las diferentes herramientas: anotaciones, medidas, y asistente de importación\slash exportación.
	\item Visor: elemento gráfico que se encarga de mostrar un modelo. Es el rectángulo negro sobre el que se ve el modelo.
\end{itemize}

\subsection{Login del usuario}
Una vez estemos autorizados a usar la aplicación (típicamente el responsable de la asignatura nos lo permitirá), lo primero que necesitamos hacer es conectarnos a la misma. Para ello, abriremos un navegador \textit{web} compatible e introduciremos la dirección apropiada. Una vez allí, nos aparecerá la ventana de \textit{login}, similar a la figura ~\ref{fig:viewer-login}. En ella vamos a introducir las mismas credenciales de acceso que empleamos para los servicios \textit{on-line} en la Universidad de Burgos. Dicha página nos redirigirá automáticamente a la sección adecuada.
\imagenpersonalizada{viewer-login}{\textit{Login} de la aplicación}{1}

\subsection{Barra de navegación}
Excepto en la página de \textit{login}, tenemos un elemento común que aparece en todas las vistas de la aplicación. Podemos aprovecharnos de ella para realizar tres acciones: ir a la vista de modelos, ir a la vista para subir éstos, y cerrar la sesión actual. Podemos apreciar dicha barra en la figura~\ref{fig:viewer-navigation-bar}.
\imagenpersonalizada{viewer-navigation-bar}{Barra de navegación}{1}

\subsection{Repositorio de modelos}
Una vez se haya conectado, lo más probable es que la primera página que vea sea la estantería de modelos. En ella podrá elegir el modelo que desea visualizar; para ello, solamente debe pinchar sobre este y automáticamente se abrirá un visor con el modelo seleccionado. Los modelos aparecerán con una miniatura para ser fácilmente identificables, tal y como se aprecia en la figura~\ref{fig:viewer-shelf}.
\imagenpersonalizada{viewer-shelf}{Repositorio de modelos}{1}

\subsection{Manipulando el modelo}
Existen diferentes acciones básicas que podemos realizar para manipular el modelo y adecuar su posición consiguiendo así una mejor perspectiva del mismo; estas acciones son rotar, mover y enfocar.

El enfoque o \textit{zoom} es realmente sencillo, solamente necesitamos mover la rueda del ratón. Para que esto suceda, el puntero debe estar ubicado dentro del visor. Como acción alternativa a esta (si nuestra rueda no funciona tan bien como quisiéramos), también podemos pinchar y mantener pulsado mediante el botón central del ratón (la rueda) mientras movemos el ratón.

La rotación requiere que pinchemos con el botón izquierdo del ratón sobre el visor, mantengamos pulsado y movamos el puntero del ratón. Cuando soltemos, el movimiento del ratón dejará de afectar al modelo.

El desplazamiento es similar a la rotación, pero se diferencia en que usa el botón derecho del ratón en vez de el izquierdo.

\subsection{Acciones avanzadas con el modelo}
Solo la visualización del modelo nos da interesantes oportunidades para conocerlo, pero tenemos disponibles un mayor número de opciones.

Viendo la figura~\ref{fig:viewer-interface} podemos observar la disposición de la interfaz.
\imagenpersonalizada{viewer-interface}{Interfaz de usuario del visor}{1}

\subsubsection{Acciones comunes}
Tanto para el menú de anotaciones como el de medidas, existen acciones comunes. Éstas son:
\begin{itemize}
	\item Añadir: este botón añade un nuevo elemento tanto en el menú adecuado como en el visor.
	\item Editar: mediante dicho botón podemos cambiar la etiqueta del mismo.
	\item Eliminar: pulsando este botón borra la presencia de dicho elemento tanto del menú correspondiente como en el visor.
	\item Seleccionar/Deseleccionar: esta acción no se lleva a cabo pulsando ningún botón; se realiza haciendo \textit{click} sobre cualquier elemento (anotación o medida) que aparezca tanto en el visor como en la lista del menú correspondiente. Sobre las medidas, se realiza pulsando sobre cualquiera de los dos puntos que la componen.
	\item Deseleccionar todo: dicho botón nos permite desmarcar todos los elementos del mismo tipo que el del menú empleado.
	\item Cambiar la escala: dentro del visor, en la parte superior derecha, tenemos un pequeño menú que nos permitirá realizar diversos ajustes con la escala del modelo, además de visualizar una malla para hacernos una idea de las dimensiones.
\end{itemize}

\subsubsection{Anotaciones}
En la figura~\ref{fig:viewer-femur-annotation-greater-trocanter} podemos apreciar dos anotaciones sobre un fémur: el <<trocánter mayor>>, seleccionado y que aparece resaltado en la lista y en verde el modelo, y el <<trocánter menor>>, que no se encuentra seleccionado.
\imagenpersonalizada{viewer-femur-annotation-greater-trocanter}{Dos anotaciones sobre un fémur}{1}

Para añadir una nueva anotación, pincharemos primero en el botón <<Añadir>> del menú de anotaciones, y después pincharemos sobre el modelo. Actualmente no se puede cancelar la acción de añadir, así que si pinchamos accidentalmente dicho botón, y no queríamos crear ninguna anotación, primero tendremos que añadir un punto cualquiera, y después eliminarlo.

Para editar una anotación, primero tenemos que seleccionar la deseada. Seleccionaremos solamente una, ya sea desde el menú o a través del modelo. Después pulsaremos sobre <<Editar>>, y nos aparecerá una ventana para cambiar la etiqueta. Una vez cambiado el texto, podremos guardar los cambios pulsando <<Save>> o descartarlos, pudiendo pinchar tanto en <<Cancel>> como cerrando la ventana.
En la figura~\ref{fig:viewer-femur-annotation-editing} podemos observar el diálogo que nos permite editar la anotación.
\imagenpersonalizada{viewer-femur-annotation-editing}{Editando una anotación}{1}

Si lo que queremos es eliminar una anotación, seleccionaremos aquellas que deseemos descartar (en este caso sí pueden ser varias). Acto seguido, pulsaremos eliminar. Si no tenemos ninguna seleccionada antes de pulsar en <<Eliminar>>, podremos borrar la siguiente anotación sobre la que pulsemos. Desafortunadamente, en este segundo caso tenemos un comportamiento similar al de añadir: hasta que no eliminemos una anotación, no podemos cancelar este estado. Una <<solución>>, si nos hemos equivocado y no queremos eliminar ninguna de las anotaciones disponibles, reside en añadir nosotros otra de forma aleatoria. Así, la creamos con el procedimiento anterior, y después de ésto, lo que haremos es pinchar otra vez sobre la misma en el visor.

Finalmente, si tenemos varias anotaciones seleccionadas y queremos que no lo estén, únicamente pincharemos en el botón <<Deseleccionar todo>> en el menú correspondiente.

\subsubsection{Medidas}
Para las medidas, los procedimientos antes descritos son similares, así que explicaremos las diferencias a continuación (además de emplear los botones de su menú).

Cuando añadimos, tenemos que pulsar sobre dos ubicaciones del modelo para conseguir finalizar la medida.

Editar se hace del mismo modo.

Para eliminar una medida, la selección se puede realizar pinchando cualquiera de las dos esferas que componen una medida.

Y por supuesto, deseleccionar todas ellas es idéntico.

\subsubsection{Importación / Exportación}
Se provee funcionalidad de importar y exportar las anotaciones y mediciones de un modelo para que un profesor pueda compartirlas y para que los alumnos puedan enviar los ejercicios a un profesor y este las corrija.

Mediante el botón de exportar, descargaremos un fichero que contiene los elementos que tengamos en el momento de invocar dicha función. Nos aparecerá un menú en el que podremos elegir tanto la ubicación como el nombre del archivo.

Si lo que necesitamos es añadir a nuestro modelo un conjunto de elementos previamente creado, utilizaremos el botón de importar. Al pulsarlo, podremos ver un menú que nos permita seleccionar el archivo en cuestión.

\subsection{Subir modelos}
Finalmente, explicaremos cómo subir modelos. Sin embargo, esta vista puede que no esté permitida para algunos usuarios.

El uso de la misma es realmente sencilla; primero iremos a esta vista pulsando en <<Subir>> en la barra de navegación, después pulsaremos el botón de seleccionar archivo en la interfaz (típicamente se llamará <<Examinar>>), y nos aparecerá una ventana en la que seleccionar el modelo. Una vez seleccionado, pulsaremos el botón <<Subir>> y esperaremos a que el proceso termine. Una vez iniciada la subida aparecerá una barra que nos indicará el progreso, y cuando finalice el mismo se nos mostrará un mensaje avisándonos del éxito. Dicha pantalla puede ser vista en la figura~\ref{fig:viewer-upload-dragon}.
\imagenpersonalizada{viewer-upload-dragon}{Diálogo de subida de modelos}{1}
