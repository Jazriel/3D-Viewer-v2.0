\capitulo{2}{Objetivos del proyecto}

Con el fin de solucionar los problemas concernientes a la docencia antes mencionados, comenzaremos por definir los resultados esperados. Necesitaremos un visor que represente gráficamente los modelos así como un sistema que sea capaz de gestionar la información que se añadirá sobre los mismos. Además, tendremos que limitar el acceso a la aplicación al público esperado mediante un sistema de validación de usuarios. Esta restricción se da para mantener los modelos virtuales seguros, de forma similar a los físicos; solamente podrán acceder a ellos los alumnos autorizados, así como profesores e investigadores a los que se desee dar acceso.

Durante la ejecución del proyecto, añadiremos como objetivo la realización de análisis sobre archivos de los modelos. Con esto, lograremos descubrir aquellos que hayan sido generados de una manera poco correcta (con puntos en el fichero que no formen parte del modelo).

\section{Visualización}
Lo primero que necesitamos es un sistema que permita visualizar los modelos\footnote{En formato PLY (avanzar a sección~\ref{sec:formato-ply}), poseen gran cantidad de detalles y son muy realistas} que el profesor provea, y queremos que éstos no se compartan fácilmente. De esta manera, necesitamos que se carguen de forma transparente al usuario, que éste no tenga en la medida de lo posible acceso a los ficheros que lo contienen. Teniendo ambas cosas en cuenta, emplearemos algún tipo de \textit{framework} que nos permita la visualización, interacción y carga de los modelos en el formato que tenemos. Un punto importante para su uso reside en evitar la necesidad de instalación, así que pensamos en realizar el desarrollo de una página \textit{web} con el fin de tener una gran portabilidad; con un navegador y conexión a Internet los usuarios pueden comenzar.

\section{Servidor}
Con el fin de dar servicio sobre los diferentes contenidos, necesitamos un sistema que gestione los elementos de que la aplicación se compone. Por ello, crearemos un servidor basado en el \textit{microframework} Flask que nos permita generar de forma dinámica los contenidos.

\section{Control de acceso}
Dado que la aplicación está destinada a alumnos del Grado en Historia y Patrimonio, además de necesitar que los modelos sean lo más privados posible, es imperativo que la aplicación tenga algún sistema de evitar accesos no autorizados. Con este fin, crearemos un control de acceso mediante \textit{login}.

Éste restringirá el uso de nuestra herramienta verificando que, en efecto, el usuario pertenezca a la comunidad universitaria de la UBU mediante una llamada a la \textit{API} de UBUVirtual, que está basada en Moodle\footnote{Tal y como aparece mencionado en \url{https://ubuvirtual.ubu.es/mod/page/view.php?id=18&lang=es}}~\cite{github:moodle}. Adicionalmente, se verificará que pertenezca a una lista de usuarios autorizados.

\section{Gestión de la información añadida}
Además del modelo, necesitamos representar la información que generaremos para los mismos. Proveeremos de los medios adecuados para representar gráficamente y realizar la persistencia de dichos elementos, que serán anotaciones y medidas.

Gráficamente, se podrán ver junto al modelo generado en forma de esferas representando los puntos donde están colocadas en el caso de las anotaciones. En el caso de las medidas, serán esferas para los extremos junto una línea que los una.

Con el fin de que esta información se pueda intercambiar fácilmente, se diseñará una estructura de datos a seguir para poder almacenarlos en forma de archivo. En nuestro caso, diseñaremos una definición de \textit{JSON} para dicha necesidad.


\section{Análisis de los modelos}\label{sec:analisis-modelos}
En el proceso de generación del modelo\footnote{Los ficheros \textit{PLY} se consiguen a partir de múltiples fotografías obtenidas con un escáner 3D, el proceso automático no es perfecto y necesita que los investigadores reparen el modelo 3D obtenido. Incluso tras la revisión manual, en muchas ocasiones siguen apareciendo puntos que no se corresponden con el modelo}, es posible que haya pequeños errores, puntos que son ruido porque se capturan por error y no pertenecen al mismo. Aunque aparecen dichos puntos en el fichero del modelo, no pertenecen a ninguno de polígonos con los que se representa la figura, son datos sin valor alguno. Si no se veían, ¿porqué entonces nos íbamos a dar cuenta de su presencia? Con estos datos, la forma estándar de calcular las dimensiones del modelo eran imprecisas, y nos basábamos en dichas dimensiones para centrar adecuadamente el modelo durante la fase de carga en pantalla.

Con el fin de conseguir una visualización más acorde al ideal, nos vimos en la necesidad de descubrir qué puntos eran los que nos molestaban para librarnos de ellos posteriormente. Dichas herramientas se pueden encontrar en el directorio \texttt{<<Resources>>} del proyecto. En el anexo <<Manual del programador>> se explica con detalle el uso de las mismas.

Nótese que esta es una herramienta suplementaria a eliminar el ruido de fondo de forma manual, como aparece documentado en~\cite{wiki:removing-background-noise}.